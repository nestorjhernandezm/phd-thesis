\section{Thesis Outline}\label{sec:intro_thesis_outline}

Based on the concepts of cooperative \ac{D2D} communications with \ac{RLNC} in cooperative wireless networks, this thesis covers two outlines. First, it was considered a research line which its main goal was to obtain network codes for cooperation in underlay \ac{D2D} cellular networks to enhance the throughput and reduce the energy consumption but also minimize the total overhead from the \ac{BS} and the mobile devices. Here, \ac{D2D} communications take place within the cellular spectrum between mobile devices in the same cell and the interference is prevented under pre-defined network planning. Second, a research line to investigate the transmission policies for mobile devices in a decentralized multihop \ac{WLAN} was considered to enhance the previously mentioned metrics by reducing the required number of transmissions to decode a batch of packets. At the end, network coded cooperation in multihop networks and \ac{D2D} communications are combined to give data services to the end-user.

\subsection{Network Code Constructions and Regimes in Cooperative D2D Cellular Networks}

Given the increase of data rates in cellular networks such as \ac{LTE-A}, we first addressed the question of when is it reasonable for a set of devices to cooperate when downloading a multicast content. The underlying reason is that there has been improvements on the cellular data rates that had approached them to the order of \ac{WLAN} data rates. Thus, in paper {[\ref{paper:paperA}]}, we investigated which are the regions where cooperation with \ac{RLNC} achieves a better perfomance than broadcast with \ac{RLNC} in terms of the throughput and energy costs. From this work, we observed that codes with high field size provided a reduced amount of transmissions translating into a high throughput but at the expense of including overhead due to the coding coefficients used in \ac{RLNC}. Thus, in paper {[\ref{paper:paperB}]}, we extended the analytical framework from paper {[\ref{paper:paperA}]}, to evaluate code constructions to optimize and reduce the total overhead from both redundant transmissions and coding coefficients. Later, we analyzed the possible \ac{D2D} cooperative cloud (cluster) sizes to observe the tradeoff since after some point, we observed that the transmissions within the clouds will increase.

\subsection{Transmission Policies in Wireless Local Area Networks}

To provide data services to a mobile device not in the range of the \ac{BS}, we considered the case of a multihop packet erasure networks that reaches the end-user through various relays with \ac{D2D} and studied the ideal transmission policies in this scenario. To achieve this, we first made a software framework with the ns-3 simulator in paper {[\ref{paper:paperC}]}. The purpose of the framework was to enable network coding simulations with the Kodo library in standard open source simulators that permit to simulate aimed for the research community. Later, in paper {[\ref{paper:paperD}]} we made an extensive set of ns-3 simulations to analyze two transmission policies and a \ac{MAC} mechanism for the devices for resource allocation and avoid interference in a multihop packet erasure network. Also, we investigated if ideal conditions existed for the devices to access the wireless medium.