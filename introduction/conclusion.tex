\section{Conclusions}\label{sec:conclusion}

This thesis addresses critical technical challenges from problems the state of the art in multicast \ac{D2D} cooperative cellular networks with \ac{RLNC} obtained during the research of the PhD studies. Based on the observed challenges, this thesis presents a network designer the conditions of when, how and how much should a set of devices cooperate to increase the perfomance of mobile networks to provide quality content to its users. Also, this thesis presents network codes that avoid the overhead trade-off from \ac{RLNC} in multicast \ac{D2D} cooperative networks for the first time. To address the challenges in the state of the art, our findings make several proposal.

To achieve at least 2X gains against using broadcast with \ac{RLNC}, we propose to utilize cooperation when the data rates in the \ac{D2D} links are at least twice than in the cellular links and when the energy cost for sending and receiving packets within clusters are at least half of the same costs in the cellular networks. However, gains are possible even with the same data rates and energy costs but depends on the packet erasure rates in each stage. To achieve the highest throughput and least energy consumption at the \ac{BS}, it is required for all the devices inside the clouds to have their connectivity with the cellular network activated. To get the least energy consumption at the devices on average, it is required that only one device in each cloud is connected to the \ac{BS} and the others of the same cloud not. However, this has the impact of reducing the throughput from the \ac{BS}. Controlling the number of connected devices to the \ac{BS} poses a trade-off between throughput and energy. We also propose to use cooperation with clusters of the same size and up to six devices per cluster for practical packet erasure rates. Other cluster sizes are possible, but they most likely reduce the metrics considered in our studies since the clouds take different times to complete in the cellular and local stages.

To avoid the trade-off in \ac{RLNC} and obtain minimum total overhead, we propose to use telescopic codes with a large portion of the coding coefficients in the binary field for broadcast and the next field for cooperation to obtain less than 3\% of total overhead. For these codes, we found that although they provide fair less overhead than \ac{RLNC}, the achieved overhead for cooperation is slightly higher than for broadcast. This occurred because the recoding operation was defined to be made with smallest field from the first hop, thus impact the total performance. Despite this, this difference is less than one or two percentual units from the total overhead.

We created a reproducible, well-tested and maintained software framework using the Kodo library and the ns-3 simulator for the research community to quickly deploy network coding simulations of standard topologies to evaluate simple heuristics in a rapid fashion, thus helping current and future protocol developers in their design process. We verified that the framework produces accurate results. To achieve 1.5-1.75X gains agains random forwarding schemes in \ac{WLAN} using cooperation with \ac{D2D}, we propose to always recode at the relays and use \ac{MAC} mechanism to avoid interference using equal access probability for the case of equal packet erasures between the source and the relays.

Besides our proposals, in the following years, future work in the areas investigated in the thesis should consider to make implementations of the proposed solutions in real devices to develop applicable protocols. For this objective, the work in papers \#[3] and \#[4] could serve as a starting point since they make a deep review of the encoding and decoding speeds of \ac{RLNC} and variant codes in portable devices, specifically the Raspberry Pi \cite{raspberrypi}, whose \ac{CPU} architecture is the same as the mobile devices proposed in this thesis. This will give key performance indicators of the potential of our solutions while allowing us to cover other aspects. Another potential area of improvement from the state of the art is to study resource allocation frameworks in \ac{LTE-A} or even 5G networks for multicast \ac{D2D} cooperative networks with \ac{RLNC} or other network codes. From a theoretical aspect, we considered in this thesis that the interference can be avoided. However, this condition could be difficult to maintain in the near future due to the data demand. Therefore, future studies could consider removing the condition of orthogonal resources for the \ac{D2D} links and study the impact in our or other metrics to have a broader perspective for future networks. The work in paper \#[5] consider this aspect and is currently under preparation for the broadcast scenario. Still, the cooperation scenario remains to be studied. In terms of the transmission policies, even though the simulation analysis allows to review simple heuristics, theoretical studies that review the optimal policies in these scenarios are required to obtain an estimate about the maximum achievable gains.

\clearpage