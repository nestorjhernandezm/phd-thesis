\section{Background}\label{sec:background}

\begin{itemize}
\item General introduction: Cellular and local networks, \ac{IoT}, \ac{M2M}, \ac{D2D}. Why Cooperation?
\item Specific introduction: Data multicasting, tools for simulation of network coding, Multihop for data dissemination.
\item Specific problems to that should be addressed by the thesis: (i) Ideal codes for data dissemination to a large cluster of devices in a centralized manner, (ii) software tools for network coding simulations, (iii) Decentralized cooperation for spatial redudancy to enhance connectivities in multihop networks with Kodo + ns-3.
\item Introduction on network coding, RLNC and how does it help to the previous.
\end{itemize}

The exponential tendency for data demand predicted by major network providers \cite{cisco2016forecast} is evident. The use of data networks have become a topic of widespread audiences. The reason has been the flourishement of online applications services from the past decade to this date such as:  Whatsapp, Viber, Telegram (voice and data); Facebook, Twitter, Snapchat, Instagram, Google+ (social networks); YouTube, Netflix, SoundCloud, Spotify (video or audio streaming); Google Drive, Dropbox, OneDrive (data storage).

% Network Coding
Introduced by Alshwede et al. \cite{ahlswede2000network}, \ac{NC} appeared as an effective technology to remove the limitations presented previously. In this work, the authors presented a new paradigm for conveying information in communication networks. Instead of following the convention of forwarding data packets from two different data flows, the packets are mixed to create new coded packets. To decode, by receiving a coded packet and knowing the original packets, it is possible to extract the missing information. This key idea let the research community know for the first time that it was possible to code on a \textit{network} basis and not only on a \textit{link} basis as conventional \ac{FEC} technologies do.

% RLNC and applications