\section{Background}\label{sec:background}
% General introduction
The exponential tendency for data demand to the \textit{zettabyte} era, predicted by major network providers \cite{cisco2016forecast,kremling2015presentation,belllabs2016report,ericsson2015report}, is present and evident. The use of cellular networks for data consumption has become a widespread topic to the general audience. The reason has been the flourishement of online applications services from the last two decades such as:  Whatsapp, Viber (voice and messaging), Facebook, Twitter, Snapchat, Instagram, Google+ (social networking), YouTube, Netflix, SoundCloud, Spotify (video or audio streaming), Google Drive, Dropbox and OneDrive (data storage). Also, these applications have been developed not only for a \ac{PC} but also mobile smartphones, tablets and phablets that support either the Android, iOS or Windows Mobile \ac{OS}. Further, content delivery networks will be carrying two-thirds of the Internet traffic by the end of 2016 \cite{cisco2016forecast}. Hence, it is expected that the data growth will continue within this market. From all these services, streaming applications that are based in multicast scenarios where a transmitter needs to serve tens, hundreds or even thousands of receivers are becoming more frequent in mobile networks such as \ac{LTE-A} or \ac{WLAN} networks such as \ac{WiFi}. These types of scenarios pose tight requirements in terms of data throughput and delay to ensure a satisfactoring \ac{QoE}.

For the network operator, techniques that can offload the service infrastructure to cope with such data load are needed in order to satisfy the increasing demand. Further, due to network capacity constraints, the end-user might not be connected to a \ac{BS} in a cellular fashion. Instead, the connectivity might be provided by other users either within the cellular spectrum or through a \ac{WLAN}. The deployment of mobile devices without cellular coverage but in a local network can potentially be decentralized. This type of deployment will require the communicating devices to (i) employ multihop communications to ensure connectivity, (ii) use control access mechanisms to avoid interference in the local network. From the devices perspective, energy consumption due to data transmissions has become a limiting factor in terms of battery life. The reason is that mobile devices perform much more internal tasks than older devices from ten years ago. Therefore, mobile network designers need to consider mechanisms and techniques that aim for high throughput and low energy consumption both at the station and the end user devices and that are able to provide data offloading from current network infrastructures.

\subsection{Cooperative Wireless Networks}
% Cooperation
The concept of cooperation in wireless networks has been investigated before \cite{fitzek2006cooperation,fitzek2007cognitive,heide2012green,fitzek2013implementation,fitzek2013mobile}. The main goal is to diminish the amount of communications resources (data rate, energy or even storage and computational power) to convey an information of common interest from a transmitter to a set of interconnected receivers in a multicast fashion. Devices connected in this way form a \textit{mobile cloud} \cite{fitzek2013mobile}. In Fig.~\ref{fig:cooperation}, it can be observed a comparison example of no cooperation and cooperation in a multicast wireless network.

\begin{figure}[ht!]
  \centering 
  \includegraphics[width=\textwidth]{introduction/figures/cooperation.pdf}
  \caption{Cooperation in Wireless Networks.}
\label{fig:cooperation}
\end{figure} 

Without cooperation, a purple content is sent to two mobile devices in a broadcast fashion. This incurs in posible a large downloading time to ensure both devices are satisfied reducing the throughput and increasing the energy consumption.  When cooperation is considered, the content now is split into smaller blue and red pieces where each of them is sent rapidly to each device. Then, the devices exploit short-range communications (dashed line) by exchanging their missing pieces. The key underlying idea is for the devices share their missing information through a faster, short-distance and reliable link which increases the total throughput and reduces the overal energy consumption. From an operator perspective, the information \textit{as a whole} is quickly disseminated into the receivers helping the \ac{BS} to offload data. At the end, the goal of making the cluster share resources is achieved. In this way, mobile clouds allow to improve the overall network performance and user experience.

\subsection{Device-to-Device Communications and Erasure Correcting Codes}
One of the key aspects to achieve the gains proposed by the cooperative approach is the short-range technologies to be considered and its parameters to guarantee a fast and reliable link. Besides \ac{WLAN} technologies such as Bluetooth or \ac{WiFi}, there has been a large interest in \ac{D2D} communications \cite{lin2013comprehensive,asadi2014survey,feng2014device,tehrani2014device}. These type of communications permit the devices to share data without going through the cellular network which keeps the idea of data offloading. In \ac{LTE-A} networks, \ac{D2D} applications services have been included \cite{3gpp2012prose} to evaluate its improvements in the \ac{QoE}.

Another aspect that is relevant for cooperation gains in multicast scenarios is channel coding. Due to the dynamics of the wireless medium, propagation conditions, noise and interference may degradate the received \ac{SINR} thus making reception unfeasible for some period of time. In the case of packet networks this leads to \textit{erasure} channels where packets are either correctly received or lost. Therefore, to protect against packet erasures, some redundancy is added through channel coding with a \ac{FEC} technique also called an erasure correcting code. These techniques are relevant to make multicast applications reliable since feedback control through \ac{ACK} packets is not possible for a large number of devices. Different erasure correcting codes might be used for reliable multicast applications. In the literature, we can fountain rateless codes such as: LT codes and Raptor codes among the most recent. These type of codes are characterized for not having a code rate since a very large number of coded symbols can be generated.    

\subsection{}
% Network Coding, RLNC and applications
Introduced by Alshwede et al. \cite{ahlswede2000network}, \ac{NC} appeared as an effective technology to remove the limitations presented previously. In this work, the authors presented a new paradigm for conveying information in communication networks. Instead of following the convention of forwarding data packets from two different data flows, the packets are mixed to create new coded packets. To decode, by receiving a coded packet and knowing the original packets, it is possible to extract the missing information. This key idea let the research community know for the first time that it was possible to code on a \textit{network} basis and not only on a \textit{link} basis as conventional \ac{FEC} technologies do. In the state of the art, mainly two types of network coding can be identified according to how the packets are mixed from: inter-session (XOR)network coding and intra-session network coding also known as \ac{RLNC} \cite{ho2006random} introduced by Ho et al. 

\subsection{Thesis Proposal and Objectives}
Based in the previous background, this thesis considers using multicast \ac{D2D} mobile clouds in cooperative wireless networks since current state of the art focuses mostly in \ac{D2D} communications based in unicast pairs. In principle, adding more users to the cloud enhances the realibility of it and minimizes the transmissions from the \ac{BS}. Still, this might increment the transmissions inside the clouds since . Describe the objectives here from Fig.~\ref{fig:proposal}.

\begin{figure}[ht!]
  \centering 
  \includegraphics[width=\textwidth]{introduction/figures/thesis-diagrams.pdf}
  \caption{State of the Art and Thesis Proposal.}
\label{fig:proposal}
\end{figure} 
