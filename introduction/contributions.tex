\section{Contribution In This Thesis}\label{sec:contributions}

\subsection{Paper A}
\textbf{On the Throughput and Energy Benefits of Network Coded Cooperation}
\textit{N\'estor J. Hern\'andez Marcano, Janus Heide, Daniel E. Lucani, Frank H.P. Fitzek}
\\  2014 IEEE Cloud Networking Conference (Cloudnet). IEEE Press, p. 138--142.
\\ Pages: 5.

\subsection*{Motivation}
The benefits of using network coded cooperation in multicast networks for enhanced throughput and reduced energy consumption have been studied in the literature before. However, all prior works assume that the short range links used to cooperate provide a faster and more reliable interface to share missing data packets in a cloud of devices. Given that the achievable rates in cellular networks with former technologies (2G, 3G) were low when compared with for example WiFi, this assumption was reasonable. However, as new emerging technologies such as LTE-A have appeared, this assumption might not be true anymore. Moreover, new proposals in LTE-A consider using D2D communications within the same frequency bands of the cellular connections. This opens the possibility that the achievable data rates for cooperation are the same or possibly less than the cellular connections. Therefore, the goal of this work is to obtain the regions where cooperative transmission scheme with network coding provides a faster throughput and a lower energy consumption than broadcast scheme with network coding.

\subsection*{Paper Content}
This work considers a system for multicasting a batch of packets using RLNC to a cloud of devices in a heterogeneous cellular network. To disseminate the batch, two transmission schemes are evaluated: Broadcast with RLNC and cooperation with RLNC. For the cooperative scheme, two phases to obtain the data packets are considered. First, all the packets are transmitted to the cloud where it only matters for a packet to arrive at least at one device. Second, the devices share turns to distribute their packets around the whole cloud. For both schemes, the distribution of the random number of transmissions required to decode the batch is calculated. This permit us to compute the average throughput and energy consumption for transmitting and decoding the batch by assigning rate and energy costs. We include an analysis of the costs by varying their respective ratios for each scheme for a wide range of packet erasure rates to observe the regions where coooperation presents a better performance than broadcast.

\subsection*{Main Results}
In this paper, we showed that a cooperative scheme with network coding provides larger throughput gains than broadcast if the data rate in the local stage doubles the cellular stage one and a large number of devices in the cloud cooperate. Moreover, if the data rates are the same (a possibility when using LTE-A), cooperation still is a preferable choice than using broadcast. For the energy consumption, cooperation is desirable if the energy cost of transmitting a packet in the local stage is the same or less than in the cellular stage. Also, the number of devices with cellular connectivity control a trade-off for the throughput and the energy. A cloud with many devices is more reliable, thus enhancing the throughput. A cloud with less devices with cellular connectivity consumes less energy since there are less devices that need to operate in both stages of the cooperation process.

\clearpage

\subsection{Paper B}
\textbf{Throughput, Energy and Overhead of Multicast Device-to-Device Communications with Network Coded Cooperation}
\textit{N\'estor J. Hern\'andez Marcano, Janus Heide, Daniel E. Lucani, Frank H.P. Fitzek}
\\  2016 Wiley Transactions on Emerging Telecommunications Technologies (former
European Transactions on Telecommunications). Special Issue: Emerging Topics in
Device to Device Communications as Enabling Technology for 5G Systems (ETT). Wiley Press, 2016. pp. 1--17
\\ Pages: 17.


\subsection*{Motivation}
% This paper presents PlayNCool, an opportunistic protocol with local optimization based on network coding to improve the throughput of a wireless mesh network. PlayNCool goal is to improve the existing routing protocols by (1) using \textit{rlnc} to transmit the coded packets, (2) recoding at relay nodes, and especially (3) exploiting a local helper to improve the link quality. In this paper we considered the problem of how much a helper should wait before starting the transmission of the coded packets.  We show that PlayNCool can provide gains of more than 3x in individual links, which translates into a large end-to-end throughput improvement, and that it provides higher gains when more nodes in the network contend for the channel at the \textit{mac} layer, making it particularly relevant for dense mesh networks.
\subsection*{Paper Content}
% We exploit the idea of selecting a local helper can improve the quality of transmission in each link and thus enhance the end--to—end service. Each link can select a subset of helper nodes to increase reliability of a link. In our approach, a relay node chooses the next hop using information derived from routing protocol (like AODV, DSDV, or B.A.T.M.A.N. and sends a coded packet to the next hop. However, the key to guaranteeing good performance is for the selected helper to \textit{play it cool} and not transmit immediately, but wait to gather some coded packets before it starts to transmit. By accumulating the coded packet in the helper, the helper will transmit more innovative coded packets to the destination.
\subsection*{Main Results}
% In this paper, we introduced a local optimization protocol using network coding being compatible to the current routing protocols for \textit{wmn}s. This protocol provides additional gains in terms of throughput and delay performance. Our protocol, called PlayNCool, provides a simple heuristic to estimate how much the helper should wait until it transmits the coded packets. Each helper calculates the waiting time from link quality information. In particular, we have understood that the link quality from sender to receiver and helper to receive are factor to provide gain to the system.
\clearpage


\subsection{Paper C}
\textbf{Getting Kodo: Network Coding for the ns-3 Simulator}
\textit{N\'estor J. Hern\'andez Marcano, Morten V. Pedersen, P\'eter
Vingelmann, Janus Heide, Daniel E. Lucani, Frank H.P. Fitzek}
\\ 2016 ACM Workshop on ns-3 (WNS3). ACM Press, p. 101--107.
\\ Pages: 7.
\subsection*{Motivation}
% This paper presents a protocol design and simulation of a locally optimized network coding protocol, called PlayNCool, for wireless mesh networks. This paper focuses on the protocol design details needed to make the PlayNCool protocol operate in the reality without changing the current network stack. We evaluated the performance of PlayNCool using ns--3 in multi-hop topologies. Our results show that the PlayNCool protocol increases the end-to-end throughput by more than two--fold and up to four--fold in our settings.
 \subsection*{Paper Content}
% In this paper we focused on the implementation details of the PlayNCool protocol. The playNCool protocol is a thin layer between MAC and IP layers. In the PlayNCool protocol, the helper accumulates the coded packets by overhearing transmissions from the source. When it accumulates a number of coded packets (less than generation size in general), it generates coded packets by recoding, i.e., by creating linear combinations of the buffered coded packets, and transmits them to the intended relay. At this point, both the source and the helper continue to transmit coded packets to the relay until the relay signals that it has all the generation size. Then, the source stops transmitting the current generation and starts transmitting the new generation. Each helper controls the number of the coded packets that should be generated by a metric called $Budget$.
\subsection*{Main Results}
% In this paper we introduced a market based network coding protocol for meshed networks called PlayNCool. This protocol exploits local helpers to increase the performance and throughput from the source to the destination and it is compatible with existing routing protocol. In ns--3 implementation, we showed that PlayNCool increases the end-to-end gain by factor of two to four folds in the wireless mesh network.
\subsection*{Own Related Publications}
This paper provides the implementation details of the protocol introduced in [\ref{paper:paperB}]
\clearpage


\subsection{Paper D}
\textbf{On Transmission Policies in Multihop Device-to-Device Communications
with Network Coded Cooperation}
\textit{N\'estor J. Hern\'andez Marcano, Janus Heide, Daniel E. Lucani, Frank H.P. Fitzek}
\\  2016 IEEE International European Wireless Conference (EW2016). IEEE Press, p. 350--354.
\\ Pages: 5.
\subsection*{Motivation}
% In this paper we study the problem of optimal use of a relay node to reduce the transmission time using network coding. More importantly, we address  focused on the effect the active transmitting neighborhood in overall gain. We show that in systems with a fair \textit{mac} mechanism, the use of a relay, in appearance of active neighbour nodes, enhances the gain up to $3.5$x in terms of throughput compared to using only the direct link. The problem is modelled as a \textit{mdp} and the results are provided comparing simple, close--to--optimal heuristics to the optimal scheme.

 \subsection*{Paper Content}
% Our problem focuses on determining the optimal transmission policy to send $M$ data packets from $S$ to $D$ with the help of $R$ and in the presence of $X-1$ active neighbors sharing the same channel. we make the following contributions:
% \begin{itemize}
% \item \textbf{Mathematical Analysis:} we model the problem as a \textit{mdp}.
% \item\textbf{Numerical Results and Comparison to Heuristics:}
% we calculate the expected completion time for different scenarios, e.g,
% different number of neighbors, different number of packets,
% different erasure probabilities of the links between source,relay, and destination. These results shows two key and counter--intuitive results.
% Finally, a comparison between the optimal results obtained by \textit{mdp} and the simulation results of PlayNCool[\ref{paper:paperB},\ref{paper:paperC}] is provided showing that PlayNCool provides a close--to—optimal solution for many scenarios.
% \end{itemize}\subsection*{Main Results}
% We proposed a Markov Decision Process model to illustrate the optimal approach to minimize the transmission time a generation of packets from a source to a destination in the presence of active neighbors by using \textit{rlnc} and a relay approach. Our results show that PlayNCool is able to achieve the close--to--optimal performance, when the number of packets is large. More importantly, we showed that using a relay in the presence of active neighbors is beneficial even if the channel from relay to destination is not better than the channel between source and destination.
% \subsection*{Own Related Publications}
% The MDP analysis for the proposed protocol in ~[\ref{paper:paperB},\ref{paper:paperC}].
\clearpage
