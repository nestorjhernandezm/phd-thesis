\section{Contribution In This Thesis}\label{sec:contributions}

\subsection{Paper A}
\textbf{On the Throughput and Energy Benefits of Network Coded Cooperation}
\textit{N\'estor J. Hern\'andez Marcano, Janus Heide, Daniel E. Lucani, Frank H.P. Fitzek}
\\  2014 IEEE Cloud Networking Conference (Cloudnet). IEEE Press, p. 138--142.
\\ Pages: 5.

\subsection*{Motivation}
The benefits of using network coded cooperation in multicast networks for enhanced throughput and reduced energy consumption have been studied in the literature before. However, all prior works assume that the short range links used to cooperate provide a faster and more reliable interface to share missing data packets in a cloud of devices. Given that the achievable rates in cellular networks with former technologies (2G, 3G) were low when compared with for example WiFi, this assumption was reasonable. However, as new emerging technologies such as \ac{LTE-A} have appeared, this assumption might not be true anymore. Moreover, new proposals in \ac{LTE-A} consider using \ac{D2D} communications within the same frequency bands of the cellular connections. This opens the possibility that the achievable data rates for cooperation are the same or possibly less than the cellular connections. Therefore, the goal of this work is to obtain the regions where cooperative transmission scheme with network coding provides a faster throughput and a lower energy consumption than broadcast scheme with network coding.

\subsection*{Paper Content}
This work considers a system for multicasting a batch of packets using \ac{RLNC} to a cloud of devices in a heterogeneous cellular network. To disseminate the batch, two transmission schemes are evaluated: Broadcast with \ac{RLNC} and cooperation with \ac{RLNC}. For the cooperative scheme, two phases to obtain the data packets are considered. First, all the packets are transmitted to the cloud where it only matters for a packet to arrive at least at one device. Second, the devices share turns to distribute their packets around the whole cloud. For both schemes, the distribution of the random number of transmissions required to decode the batch is calculated. This permit us to compute the average throughput and energy consumption for transmitting and decoding the batch by assigning rate and energy costs. We include an analysis of the costs by varying their respective ratios for each scheme for a wide range of packet erasure rates to observe the regions where coooperation presents a better performance than broadcast.

\subsection*{Main Results}
In this paper, we showed that a cooperative scheme with network coding provides larger throughput gains than broadcast if the data rate in the local stage doubles the cellular stage one and a large number of devices in the cloud cooperate. Moreover, if the data rates are the same (a possibility when using \ac{LTE-A}), cooperation still is a preferable choice than using broadcast. For the energy consumption, cooperation is desirable if the energy cost of transmitting a packet in the local stage is the same or less than in the cellular stage. Also, the number of devices with cellular connectivity control a trade-off for the throughput and the energy. A cloud with many devices is more reliable, thus enhancing the throughput. A cloud with less devices with cellular connectivity consumes less energy since there are less devices that need to operate in both stages of the cooperation process.

\clearpage

\subsection{Paper B}
\textbf{Throughput, Energy and Overhead of Multicast Device-to-Device Communications with Network Coded Cooperation}
\textit{N\'estor J. Hern\'andez Marcano, Janus Heide, Daniel E. Lucani, Frank H.P. Fitzek}
\\  2016 Wiley Transactions on Emerging Telecommunications Technologies (former
European Transactions on Telecommunications). Special Issue: Emerging Topics in
Device to Device Communications as Enabling Technology for 5G Systems (ETT). Wiley Press, 2016. pp. 1--17
\\ Pages: 17.

\subsection*{Motivation}
To increase the transmission rate of total coded packets and reduce the energy consumed by them, \ac{RLNC} is utilized to reduce the number of transmissions required to correctly receive a batch of packets. To transmit as less coded packets as possible, practical field sizes of $2^8$ or higher could be of interest. However, for a receiver to be able to decode an encoded dataset, each coded packet the coding coefficients used to create it are appended as signalling. Thus, using a very high field could incur in large signalling. If not properly designed, this might reduce the throughput and increase the energy consumption since the amount of bits for the coding coefficients could be larger than the original packet size. This presents a trade-off in the ideal code selection for data dissemination in a wireless network. Therefore, the first objective of this work was to analyze and apply new proposed codes the avoid the tight trade-off when using \ac{RLNC} in order to optimize for the throughput and energy in cooperative and broadcast networks. Also, in paper A it was assumed a single fully-connected cloud containing many devices which could be difficult to obtain in practice. Thus, the second objective for this work was to observe the effects of considering many cloud of different sizes.

\subsection*{Paper Content}
We review and analyze the application of Telescopic Codes, a recent coding scheme, that permit to obtain the best possible trade-off for minimal overhead when using various fields in the coding scheme design. Here, we defined the overhead in terms of both the average number of transmissions required to decode and the signalling from the coding coefficients. We provide a full mathematical framework to analyze the number of transmissions required to decode for Telescopic Codes, that considers \ac{RLNC} as a special case. Later, we analyze the operational trends when considering different cloud sizes. To perform this comparison, we separated the analysis for the cellular and local stage and observed which was the dominating effect.

\subsection*{Main Results}
Our proposed schemes attain less than 3\% of total mean overhead. This is fairly lower than what can be achieved with \ac{RLNC} schemes in most of the considered cases and achieving at least 1.5-2X reductions in the total overhead. For the cloud sizes: In the cellular stage, the smallest cloud contributes the most to the total transmission time. In the local stage, the biggest cloud contributes the most to the total transmission time. The homogenous cloud size is the one that provides the minimum amount of transmissions since all clouds take the same amount of time to distribute the data. Furthermore, there is an optimal number of devices per cloud for the homogenous case. Finally, we include a comparison of all our results. 

\clearpage


\subsection{Paper C}
\textbf{Getting Kodo: Network Coding for the ns-3 Simulator}
\textit{N\'estor J. Hern\'andez Marcano, Morten V. Pedersen, P\'eter
Vingelmann, Janus Heide, Daniel E. Lucani, Frank H.P. Fitzek}
\\ 2016 ACM Workshop on ns-3 (WNS3). ACM Press, p. 101--107.
\\ Pages: 7.
\subsection*{Motivation}
In previous works in the network coding literature, the C++11 Kodo library has been utilized to make real implementations of network coding protocols possible for both the research and industry communities. While network coding protocols are evaluated in a development process, the simulation stage helps to verify the mathematical analysis, rethink the modeling if observing unexpected effects or accept a design. In the research community, the ns-3 project goals are to establish an open network simulation environment for research. The ns-3 simulator provides the framework to represent standard technologies, perform debugging, code testing and documentation that eases the simulation workflow. Although there has been various initiatives to develop simulations tools in the network coding environment, most of these: (i) may be outdated in terms of maintenance and/or functionalities, and (ii) are hard to integrate with standard technologies. Up to this point, there were no network coding libraries that are well-tested and maintained to interact with equivalent network simulation environments. In this work we provided, a set of examples compliant with ns-3 using Kodo as an external library for network coding, where we verify know and expected results from the literature. The purposes of the examples are to serve the research community as starting point to make their relevant simulations in the future.

 \subsection*{Paper Content}
There are various parts in this work with specific purposes. First, we presented the theoretical aspects of the encoding, decoding and recoding operations in \ac{RLNC} and some application scenarios are mentioned. Second, we described the ns-3 examples project using Kodo from Steinwurf and give references to setup guides and tutorials for the reader. We provided the sequential steps to get a Git repository with the examples. Third, we showed how we coupled the Kodo library with ns-3. To achieve this, we introduced a coding layer in an \ac{UDP} / \ac{IP} model in ns-3. The network coding operations are implemented by the high-level Kodo C++ bindngs. These are software-wrappers that allow to manage the library in a much simpler way. Later, we describe our three simulation examples that consider two different network topologies.
% In this paper we focused on the implementation details of the PlayNCool protocol. The playNCool protocol is a thin layer between MAC and IP layers. In the PlayNCool protocol, the helper accumulates the coded packets by overhearing transmissions from the source. When it accumulates a number of coded packets (less than generation size in general), it generates coded packets by recoding, i.e., by creating linear combinations of the buffered coded packets, and transmits them to the intended relay. At this point, both the source and the helper continue to transmit coded packets to the relay until the relay signals that it has all the generation size. Then, the source stops transmitting the current generation and starts transmitting the new generation. Each helper controls the number of the coded packets that should be generated by a metric called $Budget$.
\subsection*{Main Results}
% In this paper we introduced a market based network coding protocol for meshed networks called PlayNCool. This protocol exploits local helpers to increase the performance and throughput from the source to the destination and it is compatible with existing routing protocol. In ns--3 implementation, we showed that PlayNCool increases the end-to-end gain by factor of two to four folds in the wireless mesh network.
\subsection*{Own Related Publications}
This paper provides the implementation details of the protocol introduced in [\ref{paper:paperB}]
\clearpage


\subsection{Paper D}
\textbf{On Transmission Policies in Multihop Device-to-Device Communications
with Network Coded Cooperation}
\textit{N\'estor J. Hern\'andez Marcano, Janus Heide, Daniel E. Lucani, Frank H.P. Fitzek}
\\  2016 IEEE International European Wireless Conference (EW2016). IEEE Press, p. 350--354.
\\ Pages: 5.
\subsection*{Motivation}
% In this paper we study the problem of optimal use of a relay node to reduce the transmission time using network coding. More importantly, we address  focused on the effect the active transmitting neighborhood in overall gain. We show that in systems with a fair \textit{mac} mechanism, the use of a relay, in appearance of active neighbour nodes, enhances the gain up to $3.5$x in terms of throughput compared to using only the direct link. The problem is modelled as a \textit{mdp} and the results are provided comparing simple, close--to--optimal heuristics to the optimal scheme.

 \subsection*{Paper Content}
% Our problem focuses on determining the optimal transmission policy to send $M$ data packets from $S$ to $D$ with the help of $R$ and in the presence of $X-1$ active neighbors sharing the same channel. we make the following contributions:
% \begin{itemize}
% \item \textbf{Mathematical Analysis:} we model the problem as a \textit{mdp}.
% \item\textbf{Numerical Results and Comparison to Heuristics:}
% we calculate the expected completion time for different scenarios, e.g,
% different number of neighbors, different number of packets,
% different erasure probabilities of the links between source,relay, and destination. These results shows two key and counter--intuitive results.
% Finally, a comparison between the optimal results obtained by \textit{mdp} and the simulation results of PlayNCool[\ref{paper:paperB},\ref{paper:paperC}] is provided showing that PlayNCool provides a close--to—optimal solution for many scenarios.
% \end{itemize}\subsection*{Main Results}
% We proposed a Markov Decision Process model to illustrate the optimal approach to minimize the transmission time a generation of packets from a source to a destination in the presence of active neighbors by using \textit{rlnc} and a relay approach. Our results show that PlayNCool is able to achieve the close--to--optimal performance, when the number of packets is large. More importantly, we showed that using a relay in the presence of active neighbors is beneficial even if the channel from relay to destination is not better than the channel between source and destination.
% \subsection*{Own Related Publications}
% The MDP analysis for the proposed protocol in ~[\ref{paper:paperB},\ref{paper:paperC}].
\clearpage
