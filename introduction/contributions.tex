\section{Thesis Contributions}\label{sec:contributions}

\subsection{Paper A}
\textbf{On the Throughput and Energy Benefits of Network Coded Cooperation}
\textit{N\'estor J. Hern\'andez Marcano, Janus Heide, Daniel E. Lucani, Frank H.P. Fitzek}
\\  2014 IEEE Cloud Networking Conference (Cloudnet). IEEE Press, p. 138--142.
\\ Pages: 5.

\subsection*{Motivation}
The benefits of using network coded cooperation in multicast networks for enhanced throughput and reduced energy consumption have been studied in the literature before. However, all prior works assume that the short range links used to cooperate provide a faster and more reliable interface to share missing data packets in a cloud of devices. Given that the achievable rates in cellular networks with former technologies (2G, 3G) were low when compared with for example WiFi, this assumption was reasonable. However, as new emerging technologies such as \ac{LTE-A} have appeared, this assumption might not be true anymore. Moreover, new proposals in \ac{LTE-A} consider using \ac{D2D} communications within the same frequency bands of the cellular connections. This opens the possibility that the achievable data rates for cooperation are the same or possibly less than the cellular connections. Therefore, the goal of this work is to obtain the regions where cooperative transmission scheme with network coding provides a faster throughput and a lower energy consumption than broadcast scheme with network coding.

\subsection*{Paper Content}
This work considers a system for multicasting a batch of packets using \ac{RLNC} to a cloud of devices in a heterogeneous cellular network. To disseminate the batch, two transmission schemes are evaluated: Broadcast with \ac{RLNC} and cooperation with \ac{RLNC}. For the cooperative scheme, two phases to obtain the data packets are considered. First, all the packets are transmitted to the cloud where it only matters for a packet to arrive at least at one device. Second, the devices share turns to distribute their packets around the whole cloud. For both schemes, the distribution of the random number of transmissions required to decode the batch is calculated. This permit us to compute the average throughput and energy consumption for transmitting and decoding the batch by assigning rate and energy costs. We include an analysis of the costs by varying their respective ratios for each scheme for a wide range of packet erasure rates to observe the regions where coooperation presents a better performance than broadcast.

\subsection*{Main Results}
In this paper, we showed that a cooperative scheme with network coding provides larger throughput gains than broadcast if the data rate in the local stage doubles the cellular stage one and a large number of devices in the cloud cooperate. Moreover, if the data rates are the same (a possibility when using \ac{LTE-A}), cooperation still is a preferable choice than using broadcast. For the energy consumption, cooperation is desirable if the energy cost of transmitting a packet in the local stage is the same or less than in the cellular stage. Also, the number of devices with cellular connectivity control a trade-off for the throughput and the energy. A cloud with many devices is more reliable, thus enhancing the throughput. A cloud with less devices with cellular connectivity consumes less energy since there are less devices that need to operate in both stages of the cooperation process.

\subsection*{Related Publications}
The analytical framework for this paper is largely extended in paper [1] of the co-authored papers and paper ~[\ref{paper:paperB}] for other coding schemes and scenarios.

\clearpage

\subsection{Paper B}
\textbf{Throughput, Energy and Overhead of Multicast Device-to-Device Communications with Network Coded Cooperation}
\textit{N\'estor J. Hern\'andez Marcano, Janus Heide, Daniel E. Lucani, Frank H.P. Fitzek}
\\  2016 Wiley Transactions on Emerging Telecommunications Technologies (former
European Transactions on Telecommunications). Special Issue: Emerging Topics in
Device to Device Communications as Enabling Technology for 5G Systems (ETT). Wiley Press, 2016. pp. 1--17
\\ Pages: 17.

\subsection*{Motivation}
To increase the transmission rate of total coded packets and reduce the energy consumed by them, \ac{RLNC} is utilized to reduce the number of transmissions required to correctly receive a batch of packets. To transmit as less coded packets as possible, practical field sizes of $q = 2^8$ or higher could be of interest. However, for a receiver to be able to decode an encoded dataset, each coded packet the coding coefficients used to create it are appended as signalling. Thus, using a very high field could incur in large signalling. If not properly designed, this might reduce the throughput and increase the energy consumption since the amount of bits for the coding coefficients could be larger than the original packet size. This presents a trade-off in the ideal code selection for data dissemination in a wireless network. Therefore, the first objective of this work was to analyze and apply new proposed codes the avoid the tight trade-off when using \ac{RLNC} in order to optimize for the throughput and energy in cooperative and broadcast networks. Also, in paper A it was assumed a single fully-connected cloud containing many devices which could be difficult to obtain in practice. Thus, the second objective for this work was to observe the effects of considering many cloud of different sizes.

\subsection*{Paper Content}
We review and analyze the application of \ac{TC}, a recent coding scheme proposed in \cite{heidelucani2015composite}, that permit to obtain the best possible trade-off for minimal overhead when using various fields in the coding scheme design. Here, we defined the overhead in terms of both the average number of transmissions required to decode and the signalling from the coding coefficients. We provide a full mathematical framework to analyze the number of transmissions required to decode for Telescopic Codes, that considers \ac{RLNC} as a special case. Later, we analyze the operational trends when considering different cloud sizes. To perform this comparison, we separated the analysis for the cellular and local stage and observed which was the dominating effect.

\subsection*{Main Results}
Our proposed schemes attain less than 3\% of total mean overhead. This is fairly lower than what can be achieved with \ac{RLNC} schemes in most of the considered cases and achieving at least 1.5-2X reductions in the total overhead. For the cloud sizes: In the cellular stage, the smallest cloud contributes the most to the total transmission time. In the local stage, the biggest cloud contributes the most to the total transmission time. The homogenous cloud size is the one that provides the minimum amount of transmissions since all clouds take the same amount of time to distribute the data. Furthermore, there is an optimal number of devices per cloud for the homogenous case. Finally, we include a comparison of all our results.

\subsection*{Related Publications}
A study only focusing on the benefits of \ac{TC} was presented in paper [{[1]}] but the full analytical framework with the cloud size analysis is in paper ~[\ref{paper:paperB}]. This framework covers \ac{RLNC} as a special case which was treated in paper ~[\ref{paper:paperA}]. Currently under preparation, paper [{[5]}] considers the interference effect when no frequency planning is possible only for the broadcast scenario.

\clearpage

\subsection{Paper C}
\textbf{Getting Kodo: Network Coding for the ns-3 Simulator}
\textit{N\'estor J. Hern\'andez Marcano, Morten V. Pedersen, P\'eter
Vingelmann, Janus Heide, Daniel E. Lucani, Frank H.P. Fitzek}
\\ 2016 ACM Workshop on ns-3 (WNS3). ACM Press, p. 101--107.
\\ Pages: 7.
\subsection*{Motivation}
In previous works in the network coding literature, the C++11 Kodo library from Steinwurf has been utilized to make real implementations of network coding protocols possible for both the research and industrial communities. While network coding protocols are evaluated in a development process, the simulation stage helps to verify the mathematical analysis, rethink the modeling if observing unexpected effects or accept a design. In the research community, the ns-3 project goal is to establish an open network simulation environment for research. The ns-3 simulator provides the framework to represent standard technologies, perform debugging, code testing and documentation that eases the simulation workflow. Although there has been various initiatives to develop simulations tools in the network coding environment, most of these: (i) are outdated in terms of maintenance and/or functionalities, and (ii) are hard to integrate with standard technologies. Up to this point, there were no network coding libraries that are well-tested and maintained to interact with equivalent network simulation environments such as ns-3. In this work, we provided a set of examples compliant with ns-3 using Kodo as an external library for network coding, where we verify known and expected results from the literature. The purpose of the examples is to serve the research community as the starting point to make their relevant network coding simulations with standard technologies.

\subsection*{Paper Content}
First, we presented the theoretical aspects of the encoding, decoding and recoding operations in \ac{RLNC} and some application scenarios are mentioned. Second, we described the ns-3 examples project using Kodo and give references to setup guides and tutorials for the reader. We provided the sequential steps to get a Git repository with the examples. Third, we showed how we coupled the Kodo library with ns-3. To achieve this, we introduced a coding layer in an \ac{UDP} / \ac{IP} model in ns-3. The network coding operations are implemented by the high-level Kodo C++ bindings. These are software-wrappers that allow to manage the library in a much simpler way. Later, we describe our three simulation examples that consider two different network topologies. Here, we also indicated how does Kodo interact with ns-3 through two topology helpers. Fourth, we considered an extensive set of ns-3 simulations to verify the model accuracy when compared to theoretical known results.

\subsection*{Main Results}
This papers presents ns-3 simulations based on a functional software framework that is available for the research community. The simulations show the \ac{pmf} of the distributions for the number of transmissions required to decode for different topologies and system parameters. A large number of simulations, $10^3$, were made in each scenario to get sufficient statistical results. The presented results show the simulations and analytical results match with very high accuracy.

\subsection*{Own Related Publications}
This paper provides the ground simulation setup that utilized when analysing the system in [\ref{paper:paperD}] and paper [{[2]}].

\clearpage

\subsection{Paper D}
\textbf{On Transmission Policies in Multihop Device-to-Device Communications
with Network Coded Cooperation}
\textit{N\'estor J. Hern\'andez Marcano, Janus Heide, Daniel E. Lucani, Frank H.P. Fitzek}
\\  2016 IEEE International European Wireless Conference (EW2016). IEEE Press, p. 350--354.
\\ Pages: 5.
\subsection*{Motivation}
Due to increasing data demands in upcoming technologies, a single hop will not be sufficient to reach an end-user from the source of information, given the amount of connected devices. Instead, the end-user may have connectivity through other devices which are connected to the main source in the network that could aid in conveying information to it. Thus, this work focused on reviewing cooperative based mechanisms that can help to relay data and extend connectivity using multihop topologies with \ac{D2D} communications. Also, this work considered a decentralized approach to access the medium for reducing the inherent interference in these scenarios.

 \subsection*{Paper Content}
The work considers a system composed of a single source, $N$ intermediate relays and single destination. The destination is provided connectivity to the source through the relays. The transmission process from the source through the relays towards the destination is detailed. A key point in this study was to consider two transmissions policies and a \ac{MAC} between the relays and the destination. The purpose was to review the advantage of recoding and observe if by controlling the access, some gains could be achieved.

\subsection*{Main Results}
This papers shows that 1.5-1.75X gains are possible by using recoding between the relays and destination. The study also shows that ideal access probabilities exist to reduce the required number of transmissions to decode.

\subsection*{Own Related Publications}
The simulations to analyze our model in this work, use the setup developed in ~[\ref{paper:paperC}].

\clearpage
