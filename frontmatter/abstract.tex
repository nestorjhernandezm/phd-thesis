% mainfile: ../master.tex
\chapter*{Abstract\markboth{Abstract}{Abstract}}\label{ch:Abstract}
\addcontentsline{toc}{chapter}{Abstract}
English abstract


\chapter*{Resumé\markboth{Resumé}{Resumé}}\label{ch:Resume}
\addcontentsline{toc}{chapter}{Resumé}
Danish Abstract


\chapter*{Thesis Details\markboth{Thesis Details}{Thesis Details}}\label{ch:theis_details}
\addcontentsline{toc}{chapter}{Thesis Details}

%this is a mandatory page. You can see more at http://www.phd.teknat.aau.dk/intranet/templates/
\begin{tabularx}{\textwidth}{@{}>{\bfseries}l X@{}}
  Thesis Title: & Network Coding for Cooperation in Wireless Networks\\
  Ph.D. Student: & N\'estor Javier Hern\'andez Marcano\\
  Supervisors: & Assoc.\ Prof.\ Daniel R. Lucani, Aalborg University\\
               & Dr. Janus Heide, Steinwurf ApS\\
               & Prof.\ Frank H.P. Fitzek, Aalborg University\\
\end{tabularx}
\vspace{\baselineskip}

% list the papers included in this thesis
\noindent The main body of this thesis consist of the following papers:
\begin{itemize}

  \item[{[\ref{paper:paperA}]}] N\'estor J. Hern\'andez Marcano, Janus Heide, Daniel E. Lucani, Frank H.P. Fitzek, ``On the Throughput and Energy Benefits of Network Coded Cooperation'', \emph{2014 IEEE 3rd International Conference on Cloud Networking (IEEE Cloudnet)}, pp. 138--142, 2014.

  \item[{[\ref{paper:paperB}]}] N\'estor J. Hern\'andez Marcano, Janus Heide, Daniel E. Lucani, Frank H.P. Fitzek, ``Throughput, Energy and Overhead of Multicast Device-to-Device Communications with Network Coded Cooperation'', \emph{Wiley Transactions on Emerging Telecommunications Technologies  (former European Transactions on Telecommunications). Special Issue: Emerging Topics in Device to Device Communications as Enabling Technology for 5G Systems}, pp. 1--17, 2016.

  \item[{[\ref{paper:paperC}]}] N\'estor J. Hern\'andez Marcano, P\'eter Vingelmann, Morten V. Pedersen, Janus Heide, Daniel E. Lucani, Frank H.P. Fitzek, ``Getting Kodo: Network Coding for the ns-3 Simulator'', \emph{ACM The Workshop in ns-3 (WNS3)}, pp. 101--107, 2016.

  \item[{[\ref{paper:paperD}]}] N\'estor J. Hern\'andez Marcano, Janus Heide, Daniel E. Lucani, Frank H.P. Fitzek, ``On Transmission Policies for Multihop Device-to-Device Communications with Network Coded Cooperation'', \emph{IEEE 22th International Conference on European Wireless}, pp. 350--354, 2016.

\end{itemize}

% Optionally, you can also list other publications, you have been invovled with
\noindent In addition to the main papers, the following publications have also been made:
\begin{itemize}

  \item[{[1]}] N\'estor J. Hern\'andez Marcano, Janus Heide, Daniel E. Lucani, Frank H.P. Fitzek, ``On the Overhead of Telescopic Codes in Network Coded Cooperation'', \emph{IEEE 3rd International Conference on Cloud Networking (IEEE Cloudnet)}, pp. 138--142, 2014.

  \item[{[2]}] N\'estor J. Hern\'andez Marcano, Jeppe Pihl, Janus Heide, Jeppe Krigslund, P\'eter Vingelmann, Morten V. Pedersen, Daniel E. Lucani, Frank H.P. Fitzek, ``Wurf.it: A Network Coding Reliable Multicast Streaming Solution - NS-3 Simulations and Implementation'', \emph{ACM The Workshop in ns-3: Posters, Demos and Short-Talks Session (WNS3)}, Available online at the ns-3 website: https://www.nsnam.org/workshops/wns3-2016/posters/hernandez-demo-paper.pdf, 2016.

  \item[{[3]}] N\'estor J. Hern\'andez Marcano, Chres W. S\o rensen, Juan A. Cabrera Guerrero, Simon Wunderlich, Daniel E. Lucani, Frank H.P. Fitzek, ``On Goodput and Energy Measurements of Network Coding Schemes in the Raspberry Pi'' \emph{Accepted to MDPI, Journal of Electronics. Special Issue for the Raspberry Pi (Indexed by Elsevier Scopus and Thomson-Reuters ESCI Web of Science)}.

  \item[{[4]}] Chres W. S\o rensen, N\'estor J. Hern\'andez Marcano, Juan A. Cabrera Guerrero, Simon Wunderlich, Daniel E. Lucani, Frank H.P. Fitzek, ``Easy as Pi: A Network Coding Raspberry Pi Testbed'' \emph{Accepted to MDPI, Journal of Electronics. Special Issue for the Raspberry Pi (Indexed by Elsevier Scopus and Thomson-Reuters ESCI Web of Science)}.

\end{itemize}

\noindent This thesis has been submitted for assessment in partial fulfillment of the PhD degree. The thesis is based on the submitted or published scientific papers which are listed above. Parts of the papers are used directly or indirectly in the extended summary of the thesis. As part of the assessment, co-author statements have been made available to the assessment committee and are also available at the Faculty. The thesis is not in its present form acceptable for open publication but only in limited and closed circulation as copyright may not be ensured.