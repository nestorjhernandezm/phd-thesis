% mainfile: ../master.tex
\chapter*{Abstract\markboth{Abstract}{Abstract}}\label{ch:Abstract}
\addcontentsline{toc}{chapter}{Abstract}

Mobile users through wireless networks generate today a large demand in high-quality data services that is only expected to grow dramatically in the following years. This future demand will saturate the networks as we know them today. This becomes critical in scenarios with a very high density of users, e.g. sports stadiums, where the radio resources are insufficient to deliver a good \ac{QoE}. Thus, exploiting techniques that better use the available resources by exploiting short-range communication alternatives, e.g. \ac{WiFi}, \ac{D2D} in \ac{LTE-A}, to allow cooperation between users is at the crux of delivering the necessary \ac{QoE}. In this scenario, mobile devices with any of the previous short-range communication technologies can cooperate by forming \textit{mobile clouds}. These are clusters of devices exploiting a much faster and reliable communications link to share information. \ac{NC} has proven to be an effective solution in cooperative networks since it does not require to encode and decode on a hop-by-hop basis as state of the art erasure correcting codes require since: (i) it is possible to send recoded packets to a next hop with partial data, i.e. without having decoded and (ii) it is not required to receive each packet, but just enough coded packets instead.

In this thesis we focused on the design, analysis and simulation of cooperation techniques based on \ac{RLNC} for wireless networks. First, we investigated the operational regimes where cooperative cellular networks perform better than broadcast cellular networks in terms of data rate and energy costs and also by considering different amount of cooperating devices. Two-fold gains or higher are achievable by our cooperative schemes by transmitting at least two times faster than with a broadcast scheme. We found that no more than six devices are required to be in each cloud to obtain these gains. Further, we reviewed code constructions to avoid the inherent \ac{RLNC} trade-off for the total overhead between linear dependency or signalling when using a single field size. Our research permitted us to obtain codes that achieve less than 3\% of total overhead in our reviewed scenarios, compared to at least 10\% when using \ac{RLNC}. These works indicate when and how to cooperate between the devices to offload the network, while keeping a minimum overhead. Second, we evaluated two transmission policies and a \ac{MAC} mechanism to increase the throughput and reduce the energy consumption at the devices in a two-hop decentralized \ac{WLAN} of one source, a number of relays and one destination. To achieve this, we developed a software framework that allows to analyze these policies in a simple way using the C++11 Kodo library and the ns-3 simulator. The software tool is publicly available to the research community as a major contribution. Our results show that between 50\% and 75\% gains are achievable by using the recoding feature from \ac{RLNC} and an ideal device medium access probability. Our coding schemes and techniques address key challenges in state of the art cellular and wireless local area networks to enhance the throughput and reduce the energy consumption at the cellular \ac{BS} and the mobile devices while keeping the total overhead to the bare minimum.

In the following years, the work in this area should focus on how to design and test practical implementations that show these gains, not only for current standards, but also for 5G technologies. Other aspects are to review how these metrics are affected the interference effect when orthogonal channels are not feasible with the cellular spectrum or review other metrics and code constructions to have a complete perspective of the solutions. In terms of the transmission policies, analytical works and simple derived heuristics from them to review optimal policies for mobile clouds with many devices to observe the achievable gains of these solutions.


\chapter*{Resumé\markboth{Resumé}{Resumé}}\label{ch:Resume}
\addcontentsline{toc}{chapter}{Resumé}

Mobile brugere via trådløs netværker genererer i dag en stor efterspørgsel i høj kvalitet, datatjenester, som kun forventes at vokse dramatisk i de følgende år. Denne fremtidige efterspørgsel vil mætte de netværk, som vi kender dem i dag. Dette bliver kritisk i scenarier med en meget høj koncentration af brugere, f.eks. sport stadioner, hvor radioen ressourcer er utilstrækkelige til at levere en god \ac{QoE}. Således, udnytter teknikker for bedre bruge af de tilgængelige ressourcer ved at udnytte kortrækkende alternativer kommunikation, f.eks. \ac{WiFi}, \ac{D2D} i \ac{LTE-A}, for at tillade et samarbejde mellem brugere er kernen i at levere den nødvendige \ac{QoE}. I dette scenarie, kan mobile enheder med en hvilken som helst af de tidligere kortrækkende kommunikationsteknologier samarbejde ved at danne \textit{mobile skyer}. Disse er klynger af enheder, der udnytter en meget hurtigere og pålidelig kommunikation link til at dele information. \ac{NC} har vist sig at være en effektiv løsning i kooperative netværker, da det ikke kræver at kode og afkode på en hop-af-hop grundlag som state of the art sletning korrigere koder kræver siden: (i) det er muligt at sende omkodet pakker til en næste hop med delvise data, dvs. uden at have afkodet og (ii) er det ikke nødvendigt at modtage hver pakke, men bare nok kodede pakker i stedet.

I denne afhandling vi fokuseret på design, analyse og simulering af samarbejdsteknikker baseret på \ac{RLNC} efter trådløs netværker. Først, undersøgte vi de operationelle regimer, hvor kooperative mobilnetværker udfører bedre end broadcast mobilnetværker i form af datahastigheden og energiomkostninger, og også ved at overveje forskellige mængder af samarbejdende enheder. To-fold gevinster eller højere er opnåelige ved vores kooperative ordninger ved at sende mindst to gange hurtigere end med en broadcast ordning. Vi fandt, at der ikke mere end seks enheder skal være i hver sky at få disse gevinster. Endvidere, har vi gennemgået kode konstruktioner for at undgå den iboende \ac{RLNC} trade-off for den samlede overliggende mellem lineær afhængighed eller signalering ved brug af et enkelt felt størrelse. Vores forskning tilladt os at få koder, der opnår mindre end 3\% af den samlede overhead i vores revideret scenarier, sammenlignet med mindst 10\% ved anvendelse \ac{RLNC}. Disse værker viser, hvornår og hvordan man samarbejder mellem enhederne at losse netværket, og samtidig holde et minimum overliggende. For det andet, vi evaluerede to transmissions politikker og en \ac{MAC} mekanisme til at øge throughput og reducere energiforbruget på enhederne i en to-hop decentraliseret \ac{WLAN} af en kilde, en række relæer og én destination. For at opnå dette, har vi udviklet et framework der gør det muligt at analysere disse politikker på en enkel måde ved hjælp af C++11 Kodo bibliotek og ns-3 simulator. Den software værktøj er offentligt tilgængelig for forskersamfundet som et vigtigt bidrag. Vores resultater viser, at mellem 50\% og 75\% gevinster kan opnås ved hjælp af omkodning funktionen fra \ac{RLNC} og en ideel enhed medium adgang sandsynlighed. Vores kodningssystemer og teknikker løse centrale udfordringer i state of the art cellulære og trådløs lokalnet at forbedre gennemløb og reducere energiforbruget på cellulære \ac{BS} og de mobile enheder, samtidig med at den samlede overliggende til et absolut minimum.

I de følgende år, bør arbejdet på dette område fokus på, hvordan at designe og teste praktiske implementeringer, der viser disse gevinster, ikke kun for de nuværende standarder, men også for 5G teknologier. Andre aspekter er at gennemgå, hvordan disse målinger er påvirket interferens effekt, når ortogonale kanaler er ikke muligt med den cellulære spektrum eller gennemgå andre målinger og kode konstruktioner at have en komplet perspektiv af løsningerne. Med hensyn transmissions politikker, analytiske værker og simple afledte heuristik fra dem at gennemgå optimale politikker til mobile skyer med mange enheder til at observere de opnåelige gevinster ved disse løsninger.
