% mainfile: ../master.tex
\chapter*{Abstract\markboth{Abstract}{Abstract}}\label{ch:Abstract}
\addcontentsline{toc}{chapter}{Abstract}

Communication technologies through wireless networks generate today a large demand in high-quality data services that is only expected to grow dramatically in the following years. This future demand will saturate the networks as we know them today. Therefore, This becomes critical in scenarios with a very high density of users, e.g. sports stadiums, where the radio resources are insufficient to deliver a good \ac{QoE}. Thus, exploiting techniques that better use the available resources by exploiting short-range communication alternatives, e.g. \ac{WiFi}, \ac{D2D} in \ac{LTE-A}, to allow cooperation between users is at the crux of delivering the necessary \ac{QoE}. In this scenario, mobile devices with any of the previous short-range communication technologies can cooperate by forming \textit{mobile clouds}. These are cluster of devices exploiting a much faster and reliable communications link to share information. \ac{NC} has proven to be an effective solution in cooperative networks since it does not require to encode and decode on a hop-by-hop basis as other erasure correcting codes from the state of the art require to do. 

In this thesis we focused on the design, analysis and simulation of cooperation techniques based on \ac{RLNC} for wireless networks. First, we investigated the operational regimes where cooperative cellular networks perform better than broadcast cellular networks in terms of data rate and energy costs and also by considering different amount of cooperating devices. Two-fold gains or higher are achievable by our cooperative schemes by transmitting at least two times faster than with a broadcast scheme. We found that no more than six devices are required to be in each cloud to obtain these gains. Further, we reviewed code constructions to avoid the inherent \ac{RLNC} trade-off for the total overhead between linear dependency or signalling when using a single field size. Our research permitted us to obtain codes that achieve less than 3\% of total overhead in our reviewed scenarios. These works indicate when and how to cooperate between the devices to offload the network, while keeping a minimum overhead. Second, we evaluated two transmission policies and a \ac{MAC} mechanism to increase the throughput and reduce the energy consumption at the devices in a two-hop decentralized \ac{WLAN} of one source, a number of relays and one destination. To achieve this, we developed a software framework that allows to analyze these policies in a simple way using the C++11 Kodo library and the ns-3 simulator. The software tool is publicly available to the research community as a major contribution. Our results show that between 50\% and 75\% gains are achievable by using the recoding feature from \ac{RLNC} and an ideal device medium access probability. Our coding schemes and techniques address key challenges in state of the art cellular and wireless local area networks to enhance the throughput and reduce the energy consumption at the cellular \ac{BS} and the mobile devices while keeping the total overhead to the bare minimum.

In the following years, the work in this area should focus on how to design and test practical implementations that show these gains, not only for current standards, but also for 5G technologies. Other aspects are to review how these metrics are affected the interference effect when orthogonal channels are not feasible with the cellular spectrum or review other metrics and code constructions to have a complete perspective of the solutions. In terms of the transmission policies, analytical works and simple derived heuristics from them to review optimal policies for mobile clouds with many devices to observe the achievable gains of these solutions.


\chapter*{Resumé\markboth{Resumé}{Resumé}}\label{ch:Resume}
\addcontentsline{toc}{chapter}{Resumé}
Danish Abstract
