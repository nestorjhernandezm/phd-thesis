% mainfile: ../master.tex
\chapter*{Abstract\markboth{Abstract}{Abstract}}\label{ch:Abstract}
\addcontentsline{toc}{chapter}{Abstract}

Communication technologies through wireless networks generate today a large demand in high-quality data services that is only expected to grow dramatically in the following years. This future demand will saturate the networks as we know them today. Therefore, cellular network operators are interested in techniques that can offload this traffic to \ac{WLAN} such as \ac{WiFi}, or more recently \ac{D2D} communications. In this scenario, mobile devices that have any of the previous short-range communication technologies can cooperate by exploiting a much faster and reliable connectivity link to share information. Here, \ac{NC} and particularly \ac{RLNC} has proven to be an effective solution to this scenario since it possess most of the advantages of state of the art \ac{FEC} techniques but it is also well-suited for the cooperative case, while these \ac{FEC} are not.

In this thesis we focused on the design, analysis and simulation of cooperation techniques based on \ac{RLNC} for wireless networks. First, we investigated the operational regimes where cooperative cellular networks perform better than broadcast cellulars networks in terms of data rate and energy costs and also by considering different amount of devices cooperation. Further, we reviewed code constructions that are optimal in other key performance metrics such as total overhead. These works indicate when and how to cooperate between the devices to offload the network, while keeping a minimum overhead Second, we evaluated transmission policies and \ac{MAC} mechanisms to reduce the energy consumption at the devices and increase the throughput in decentralized in \ac{WLAN}. Here, we developed a software framework to perform our analysis using the ns-3 simulator that is publicly available to the research community as a major contribution and showed the possibility to analyze heuristics in an easy fashion. Overall, our techniques address key challenges in state of the art wireless networks to cope with future demands in a satisfactory manner for operator and end-users. 


\chapter*{Resumé\markboth{Resumé}{Resumé}}\label{ch:Resume}
\addcontentsline{toc}{chapter}{Resumé}
Danish Abstract
